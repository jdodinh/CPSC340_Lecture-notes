% ---
\documentclass{article}


% Packages
% ---
\usepackage{amsmath,amssymb,amsthm} 	% Advanced math typesetting
% \usepackage[utf8]{inputenc} 	% Unicode support (Umlauts etc.)
\usepackage[USenglish]{babel} 	% Change hyphenation rules
\usepackage{hyperref} 				% Add a link to your document
\usepackage{graphicx}				% Add pictures to your document
\graphicspath{ {./images/} }	% image directory
\usepackage{listings} 				% Source code formatting and highlighting
\lstset{basicstyle=\ttfamily}		%Typewriter font for code writing
\usepackage{geometry}
\geometry{margin=1in}
\usepackage{enumitem}
\usepackage{float}
%\floatstyle{boxed}
\restylefloat{figure}
\usepackage{mathabx}
\usepackage{fancyhdr}
\usepackage[dvipsnames]{xcolor}
\usepackage{scrextend}

% Colors
\definecolor{blu}{rgb}{0,0,1}
\def\blu#1{{\color{blu}#1}}
\definecolor{gre}{rgb}{0,.5,0}
\def\gre#1{{\color{gre}#1}}
\definecolor{red}{rgb}{1,0,0}
\def\red#1{{\color{red}#1}}
\def\norm#1{\|#1\|}

\theoremstyle{definition}
\newtheorem{definition}{Def}

\pagestyle{fancy}
\fancyhf{}
\lhead{\bf CPSC 340 \\ Week 5 }
\rhead{\bf Jeremi Do Dinh \\ 61985628}
\rfoot{Page \thepage}



\usepackage{tikz}						% Graph drawing tools
\usetikzlibrary {positioning}

\usepackage{breqn}
\usepackage{multicol} 				% Multiple column functionality
\usepackage{blindtext}
\newcommand{\centerfig}[2]{\begin{center}\includegraphics[width=#1\textwidth]{#2}\end{center}}


\begin{document}

\noindent \url{https://www.cs.ubc.ca/~fwood/CS340/}

\section*{Lecture XIII}
\textbf{February 3rd, 2020} \\
\url{https://www.cs.ubc.ca/~fwood/CS340/lectures/L13.pdf}

\subsection*{Normal Equations}
To find the normal equations, we \gre{set the gradient equal to 0}, to find the critical points:
\begin{align*}
	X^TXw - X^Ty &= 0 \\
	\intertext{We now move the terms not involving $ w $ to the other side:}
	X^TXw &= X^Ty
\end{align*} 
This is a set of $ d $ linear equations, called the \blu{normal equations}. In \blu{Python} we sole linear equations in one line, using \texttt{numpy.linalg.solve}.

\subsection*{Least squares cost}
We investigate the cost of solving normal equations $ X^TXw = X^Ty $:
\begin{itemize}
	\item Forming $ X^Ty $ vectors costs $ O(nd) $
	\begin{itemize}
		\item It has $ d $ elements, and each is an inner product between $ n $ numbers.
	\end{itemize}
\item Forming the matrix $ X^TX $ costs \red{$ O(nd^2) $}
\begin{itemize}
	\item It has $ d^2 $ elements, and each is an inner product between $ n $ numbers.
\end{itemize}
\item Solving a $ d \times d $ system of equations costs \red{$ O(d^3) $}
\begin{itemize}
	\item Cost of Gaussian elimination on a d-variable linear system.
	\item Other standard methods have the same cost
\end{itemize}
\item Therefore the overall cost is $ \blu{O(nd^2 + d^3)} $.
\begin{itemize}
	\item The dominating term depends on $ n $ and $ d $
\end{itemize}
\end{itemize}

\subsection*{Least squares issues}
\begin{itemize}
	\item Solution \red{might not be unique}.
	\item It is \red{sensitive to outliers}.
	\item It always \red{uses all features}.
	\item Data can might so big we \red{can’t store $ X^TX $}.
	\begin{itemize}
		\item Or you can’t afford the $ O(nd^2 + d^3) $ cost.
	\end{itemize}
	\item It might \red{predict outside range} of yi values.
	\item It assumes a \red{linear relationship} between xi and yi.
\end{itemize}
 \subsection*{Non-Uniqueness of Least Squares Solution}
Why isn't the solution unique?
\begin{itemize}
	\item We investigate the case where we have \gre{two features that are identical} for \textit{all} examples. 
	\item We can increase weight on one feature, and decrease it on the other, \red{without changing predictions}.
	\centerfig{0.6}{Pic1}
	\item Thus, if $ (w_1,w_2) $ is a solution then $ (w_1+w_2, 0) $ is another solution.
	\item This is special case of features being \blu{collinear}:
\end{itemize}
But any $ w $ where $ \nabla f(w)=0 $ is a global minimizer of $ f $.

\end{document}